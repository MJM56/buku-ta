\begin{center}
  \large\textbf{ABSTRAK}
\end{center}

\addcontentsline{toc}{chapter}{ABSTRAK}

\vspace{2ex}

\begingroup
% Menghilangkan padding
\setlength{\tabcolsep}{0pt}

\noindent
\begin{tabularx}{\textwidth}{l >{\centering}m{2em} X}
  Nama Mahasiswa    & : & Batrisyia Zahrani Ananto        \\

  Judul Tugas Akhir & : &     KONTROL PERGERAKAN KURSI RODA DENGAN \emph{HEAD GESTURE} MENGGUNAKAN MEDIAPIPE BERBASIS NUC      \\

  Pembimbing        & : & 1. Dr. Eko Mulyanto Yuniarno,S.T.,M.T.   \\
                    &   & 2. Dion Hayu Fandiantoro, S.T.,M.Eng. \\
\end{tabularx}
\endgroup

% Ubah paragraf berikut dengan abstrak dari tugas akhir
Kuadriplegia atau tetraplegia merupakan kelumpuhan yang terjadi pada keempat anggota gerak tubuh. Karena limitasi yang mereka miliki, maka mereka memerlukan kursi roda. Mengendalikan pergerakan kursi roda bisa menjadi tantangan, terutama bagi pengguna dengan keterbatasan fisik signifikan. Diperlukan pengembangan sistem kontrol kursi roda yang dapat digunakan penderita tetraplegia. Salah satu pendekatan yang dapat dilakukan untuk mengontrol kursi roda adalah dengan melakukan ekstraksi fitur wajah dan pendeteksian \emph{head gesture} dengan menggunakan mediapipe. Kemudian data \emph{head gesture} yang telah diklasifikasikan akan dikirimkan oleh NUC ke sistem kontrol kursi roda. Sistem kontrol tersebutlah yang akan mengatur arah gerak kursi roda. Dengan menggunakan metodologi yang digunakan, dapat ditarik beberapa kesimpulan dari pengujian yang telah dilakukan. Model yang akan digunakan memiliki arsitektur CNN 7 layer dengan Convolutional 2D 64, 256 dan diakhiri dengan Dense 512. Jarak model yang paling tinggi akurasinya adalah 50 sentimeter. Intensitas cahaya yang paling tinggi akurasinya adalah 110 lux. Kecepatan FPS laptop penulis lebih tinggi daripada NUC yang digunakan. Rata-rata waktu \emph{delay} renspons motor adalah 0,3025423729 detik dan \emph{inference time}nya adalah 0,07220 detik. Rata-rata kestabilan gerak motor kursi roda untuk pendeteksian selama 2 detik adalah 8,9364 detik.

% Ubah kata-kata berikut dengan kata kunci dari tugas akhir
Kata Kunci: Gestur Kepala, Kursi Roda, Kontrol , NUC, Tetraplegia.
